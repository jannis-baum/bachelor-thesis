% English
\null\vfil
\begin{otherlanguage}{english}
\begin{center}\textsf{\textbf{\abstractname}}\end{center}

    \noindent Reliably and responsibly maintaining health-related informative
    resources requires human supervision in data administration. One such
    resource is PharMe: a service aiming at bringing \gls{pgx} findings closer
    to patients to accelerate its adoption in their treatment. In this thesis,
    I examine PharMe's approach to data administration and discuss the related
    complications, namely its high communication complexity, susceptibility to
    human error, and inaccessibility to the field experts who supervise it.
    Subsequently, I conceptualize, implement and test a method of solving these
    complications, and to create the fundamental infrastructure towards
    providing information with support for multiple languages. The presented
    method achieves this by interactively modularizing information, and thereby
    also increases its consistency and efficiency in maintenance and initial
    curation. The resulting findings facilitate the process of data
    administration for PharMe while providing relevant insights into its future
    development and towards establishing solutions to similar problems in other
    health-related informative resources.

    % In general, good structure! However, you can be more specific with stating your
    % findings: the abstract is rather “the thesis in short” than a summary of the
    % thesis topics.

    % Also, you could give an even stronger motivaton and problem statement, which can
    % resemble your motivation in the Introdcution, e.g.,
    % "PGx and its potential advantages"
    % > "need to be communicated to be useful"
    % > "curating texts is difficult/time intensive to do"
    % > "this is improved by the Annotation Interface" (do you have a catchy name for
    % your tool?)

    % Then explain how you implemented it, how it was perceived in the testing, and what
    % your conclusions are :)

\end{otherlanguage}
\vfil\null

% German
\null\vfil
\begin{otherlanguage}{ngerman}
\begin{center}\textsf{\textbf{\abstractname}}\end{center}

    \noindent to do.

\end{otherlanguage}
\vfil\null
