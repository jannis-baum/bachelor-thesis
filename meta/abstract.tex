% English
\null\vfil
\begin{otherlanguage}{english}
\begin{center}\textsf{\textbf{\abstractname}}\end{center}

    \noindent PharMe is a service that aims to bring \glsa{pgx} and its
    potential advantages to the treatment of patients by providing them with
    personalized \gls{pgx} insights. To be useful to patients, these insights
    need to be communicated comprehensively. Researching, curating and
    administering the necessary information is a time intensive process
    conducted by \glsa{pgx} experts. This process is facilitated by the
    Annotation Interface researched and implemented in this thesis. The primary
    used method modularizes information to be made up of predefined textual
    components rather than free text. This increases consistency,
    maintainability, and curation efficiency of information, while also laying
    the foundation for support of multilingualism: By merely translating its
    components, all current and future information is available in the
    respective language. The presented methods attend PharMe's need for
    administration of multilingual information and provide insights towards
    establishing systems to solve similar problems for other health-related
    informative resources.

\end{otherlanguage}
\vfil\null

% German
\null\vfil
\begin{otherlanguage}{ngerman}
\begin{center}\textsf{\textbf{\abstractname}}\end{center}

    \noindent PharMe ist ein Service mit dem Ziel, PGx und seine potenziellen
    Vorteile in die Behandlung von Patienten einzubringen, indem er ihnen
    personalisierte pharmakogenomische Erkenntnisse liefert. Um für die
    Patienten nützlich zu sein, müssen diese Erkenntnisse verständlich
    vermittelt werden. Das Recherchieren, Zusammenstellen und Verwalten der
    erforderlichen Informationen ist ein zeitintensiver Prozess, der von
    PGx-Experten durchgeführt wird. Dieser Prozess wird durch das Annotation
    Interface, das mit dieser Arbeit erforscht und implementiert wird,
    erleichtert. Die primär verwendete Methode modularisiert die Informationen,
    so dass sie aus vordefinierten Textkomponenten anstatt aus freiem Text
    bestehen. Dies erhöht die Konsistenz und Effizienz der Informationspflege
    und schafft gleichzeitig die Grundlage für die Unterstützung mehrerer
    Sprachen: Bereits die Übersetzung der Komponenten genügt, um alle aktuellen
    und zukünftigen Informationen in der jeweiligen Sprache zur Verfügung zu
    stellen. Die vorgestellten Methoden erfüllen PharMe's Erfordernis an der
    Verwaltung mehrsprachiger Informationen und bieten Einblicke in die
    Entwicklung von Systemen zur Lösung ähnlicher Probleme für andere
    gesundheitsbezogene Informationsressourcen.

\end{otherlanguage}
\vfil\null
