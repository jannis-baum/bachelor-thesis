% English
\null\vfil
\begin{otherlanguage}{english}
\begin{center}\textsf{\textbf{\abstractname}}\end{center}

    \noindent Reliably and responsibly maintaining health-related informative
    resources requires human supervision in data administration. One such
    resource is PharMe: a service aiming at bringing \gls{pgx} findings closer
    to patients to accelerate its adoptation in general physicians' consulting
    rooms. In this thesis, I examine PharMe's preexisting approach to data
    administration and discuss the related complications, namely its high
    communication complexity and susceptibility to human error. Subsequently, I
    conceptualize, implement and test a method of solving these complications,
    and to create the fundamental infrastructure towards providing information
    with support for multiple languages. The presented method primarily
    achieves this by modularizing information, and thereby also increases its
    consistency and efficiency in maintenance and initial curation. The
    resulting findings facilitate the process of data administration for PharMe
    while providing relevant insights into its future development and towards
    establishing solutions to similar problems in other health-related
    informative resources.

\end{otherlanguage}
\vfil\null

% German
\null\vfil
\begin{otherlanguage}{ngerman}
\begin{center}\textsf{\textbf{\abstractname}}\end{center}

    \noindent Zuverlässige und verantwortungsvolle Bereitstellung
    gesundheitsbezogener Informationsressourcen erfordert menschliche Aufsicht
    bei der Datenverwaltung. Eine solche Ressource ist PharMe: ein Dienst mit
    dem Ziel, pharmakogenomische Erkenntnisse näher an Patient*innen zu bringen
    und dessen Adoption in der ärztlichen Behandlung zu beschleunigen.  In
    dieser Arbeit untersuche ich die von PharMe bereits bestehenden Ansätze zur
    Datenverwaltung und diskutiere ihre Problematik der hohen
    Kommunikationskomplexität und der Anfälligkeit für menschliche Fehler.
    Anschließend konzipiere, implementiere und teste ich eine Methode zur
    Lösung der Problematik und zur Schaffung einer grundlegenden Infrastruktur
    für die Bereitstellung der Informationen mit Unterstützung für mehrere
    Sprachen. Die vorgestellte Methode erreicht dies vorrangig durch die
    Modularisierung von Informationen und steigert dadurch außerdem ihre
    Konsistenz und die Effizienz der Pflege und initialen Kuratierung. Die sich
    daraus ergebenden Erkenntnisse erleichtern den Prozess der Datenverwaltung
    für PharMe und bieten gleichzeitig relevante Einblicke in die zukünftige
    Entwicklung des Systems und in die Lösungen ähnlicher Probleme in anderen
    gesundheitsbezogenen Informationsressourcen.

\end{otherlanguage}
\vfil\null
